\documentclass[letterpaper,11pt]{article}

\usepackage{latexsym}
\usepackage[empty]{fullpage}
\usepackage{titlesec}
\usepackage{marvosym}
\usepackage[usenames,dvipsnames]{color}
\usepackage{verbatim}
\usepackage{enumitem}
\usepackage[hidelinks]{hyperref}
\usepackage{fancyhdr}
\usepackage[english]{babel}
\usepackage{tabularx}
\usepackage{res}
\usepackage{fontawesome5}
\usepackage{csquotes}
\input{glyphtounicode}

%for the bibliography
\usepackage[
    style=apa,
    sortcites=true,
    backend=biber,natbib,
    labeldate=year
]{biblatex}
\addbibresource{publications.bib}
% Ensure only the year is displayed in citations & references
\AtEveryBibitem{\clearfield{month}\clearfield{day}\clearfield{date}}
\AtEveryCitekey{\clearfield{month}\clearfield{day}\clearfield{date}}

\begin{document}

%----------HEADING----------
\begin{tabular*}{\textwidth}{l@{\extracolsep{\fill}}r}
  \textsc{\href{https://huwcheston.github.io/}{\Large Huw Cheston}} & \faMap \ Location: \textsc{London, UK} \\ 
  \href{https://huwcheston.github.io/}{\faGlobe \ Website: https://huwcheston.github.io/} & \faEnvelope \ Email: hwc31@cam.ac.uk \\  \href{https://github.com/huwcheston}{\faGithub \ GitHub: https://github.com/huwcheston} & \faPhone \ Phone: +44 7474 654583 \\
\end{tabular*}

% \href{https://scholar.google.com/citations?user=byas-BIAAAAJ\&hl=en}{Google Scholar}

%-----------EXPERIENCE-----------
\section{Experience}
  \resumeSubHeadingListStart

    % \resumeSubheading
    %   {Data Scientist}{June 2025 --}
    %   {Colossal.FM (Part-Time)}{Remote}
    %   \resumeItemListStart
    %     \resumeItem{}
    %   \resumeItemListEnd

    \resumeSubheading
      {Research Assistant, Multimodal AI}{May 2025 --}
      {Queen Mary University of London (Part-Time)}{London, UK}
      \resumeItemListStart
        \resumeItem{Conducted research into multimodal machine perception \textbf{in collaboration with Meta} (\href{https://www.projectaria.com/}{Project Aria}).}
        \resumeItem{Developed machine-learning models that combine vision and audio to produce representations of real-world scenes.}
      \resumeItemListEnd

    \resumeSubheading
      {Research Scientist Intern, Audio Intelligence}{June -- August 2024}
      {Spotify}{London, UK}
      \resumeItemListStart
        \resumeItem{Developed end-to-end deep learning model for identifying copyrighted samples in catalogue audio recordings, \textbf{driving downstream product applications} in music content categorisation \& plagiarism detection}
        \resumeItem{Results \textbf{outperformed competing system (``Shazam'') by 9x}, improved upon internal model by \textbf{13\%}}
        \resumeItem{Managed pipelines for generating artificial training data at \textbf{petabyte-scale} using Google Cloud \& Apache Beam}
        \resumeItem{Deployed training runs on distributed GPU clusters using Ray \& Kubernetes}
        \resumeItem{\textbf{Presented results} to senior company stakeholders and in a scientific paper \href{https://arxiv.org/abs/2502.06364}{[accessible at arXiv:2502.06364]}}
      \resumeItemListEnd

    \resumeSubheading
      {Music Data Science Lecturer \& Supervisor}{2021 -- 2024}
      {University of Cambridge}{Cambridge, UK}
      \resumeItemListStart
      \resumeItem{Delivered \textbf{100+} undergraduate supervisions and lectures on  modelling and visualisation of audio data}
      \resumeItem{\textbf{Managed} 4 undergraduate students on music-related data science and machine learning projects}
    \resumeItemListEnd

    % \resumeSubheading
    %   {Data Science Instructor}{2023}
    %   {Sutton Trust}{Cambridge, UK}
    %   \resumeItemListStart
    %     \resumeItem{Delivered workshops on music + data science for secondary-age students from state-educated backgrounds}
    %     \resumeItem{Designed interactive coding and statistics exercises on Google Colab, hosted on GitHub Pages}
    % \resumeItemListEnd

    \resumeSubheading
      {Teaching Assistant}{2020 -- 2021}
      {Kingswood School}{Bath, UK}
      \resumeItemListStart
      \resumeItem{Planned \& taught music technology lessons, both in-classroom and virtually during COVID}
      \resumeItem{Managed recording studio and music technology suite, produced promotional audio-visual material for the school}
    \resumeItemListEnd

    \resumeSubheading
      {Professional Musician}{2016 -- 2020}
      {Freelance}{UK}
      \resumeItemListStart
      \resumeItem{Worked with internationally recognised acts including \textit{Clean Bandit}, \textit{Everything Everything}, \& \textit{Dinosaur}.} 
      \resumeItem{Performed regularly around the UK \& abroad, in front of crowds exceeding 30,000+ capacity.}
    \resumeItemListEnd
  \resumeSubHeadingListEnd

%-----------EDUCATION-----------
\section{Education}
  \resumeSubHeadingListStart
    \resumeSubheading
      {University of Cambridge}{2021 -- 2025}
      {Ph.D, Music Computing}{Cambridge, UK}
      \resumeItemListStart
        % comment out for "professional"/non-academic
        % \resumeItem{\textbf{Thesis title}: Computational Modelling of Jazz Improvisation}
        \resumeItem{\textbf{Advanced state-of-the-art} in music computing tasks including automatic artist identification from audio}
        \resumeItem{Published \textbf{5 first author} papers with several in major (top-20) science \& computer science journals}
        \resumeItem{Fully-funded with £80k competitive research grant from Cambridge Trust}
        % comment out for "professional"/non-academic
        \resumeItem{\textbf{Service:} peer review for \textit{Transactions of the International Society of Music Information Retrieval}, \textit{Music \& Science} journals, \textit{International Society for Music Information Retrieval} conference}
      \resumeItemListEnd
    \resumeSubheading
      {University of Oxford}{2016 -- 2020}
      {BA + MSt., Music}{Oxford, UK}
      \resumeItemListStart
        \resumeItem{Graduated with \textbf{highest mark in year}, 85\% average}
        \resumeItem{Fully-funded masters study with £25k research grant from Linacre College, Oxford}
      \resumeItemListEnd
  \resumeSubHeadingListEnd

%-----------SKILLS-----------
\section{Skills}
 \begin{itemize}[leftmargin=0.15in, label={}]
    \small{\item{
     \textbf{Stats}{: multi-level modelling, NHST workflows, time-series analysis, experiment design, optimisation} \\
     \textbf{Inference}{: simulation, AB testing, hypothesis testing, bootstrapping, causal inference, dimensionality reduction,} \\
     \textbf{Machine Learning}{: explainability, generative modelling, artificial datasets, neural networks, big data} \\
     \textbf{Languages}{: Python, JavaScript, HTML/CSS, R, SQL} \\
     \textbf{Tools}{: Git, Unix, \LaTeX, Google Cloud Platform, Apache Beam, Docker, Ray, MLFlow} \\
     \textbf{Libraries}{: pandas, matplotlib, scipy, scikit-learn, pytorch, statsmodels, huggingface, numpy, librosa, captum, open-ai} \\
     \textbf{Domain Knowledge}{: music categorisation \& retrieval, audio signal processing, natural language processing} \\
    \textbf{Audio Production}{: ambisonics, binaural audio, multichannel setups, digital audio workstations}
    }}
 \end{itemize}

%-----------OPEN SOURCE-----------
\section{Open Source Contributions}
 \begin{itemize}[leftmargin=0.15in, label={}]
    \small{
    \item{
    \href{https://github.com/mir-dataset-loaders/mirdata}{\texttt{mirdata}} (190k+ installs) major feature development and optimisations to data architecture for a Python library that standardizes access to audio datasets commonly used in machine learning research and development
    }
    \item{
    \href{https://github.com/Natooz/MidiTok}{\texttt{miditok}} (150k+ installs) bug fixes and optimisations for a music tokenization Python library used in generative AI music
    }}
\end{itemize}

%-----------PUBLICATIONS-----------
% comment out for "professional"/non-academic
\section{Selected Publications}
\nocite{*}
  \resumeSubHeadingListStart
    \resumeSubheading{Journal Articles}{}{}{}
    \printbibliography[heading=none, keyword=paper]
    \resumeSubheading{Preprints}{}{}{}
    \printbibliography[heading=none, keyword=preprint]
    \resumeSubheading{Conference Proceedings}{}{}{}
    \printbibliography[heading=none, keyword=proceedings]
\resumeSubHeadingListEnd

%-------------------------------------------
\section{References Available on Request}
\end{document}
